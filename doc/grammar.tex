\section{Grammar}


%HEVEA \begin{latexonly}
\newcommand{\rulelabel}[1]{%
  \parbox[b]{\labelwidth}{\makebox[0pt][l]{\hypertarget{#1}{\textit{#1}}:}}}
\newcommand{\ruleoneoflabel}[1]{%
  \parbox[b]{\labelwidth}%
  {\makebox[0pt][l]{\hypertarget{#1}{\textit{#1}}: one of}}}
\newenvironment{rules}[1]{%
  \begin{list}{}{}%
    \renewcommand{\makelabel}{\rulelabel}%
  \item[#1]\mbox{}\\}{\end{list}}
\newenvironment{rulesoneof}[1]{%
  \begin{list}{}{}%
    \renewcommand{\makelabel}{\ruleoneoflabel}%
  \item[#1]\mbox{}\\}{\end{list}}
%HEVEA \end{latexonly}
%HEVEA \newcommand{\rulelabel}[1]{%
%HEVEA   \parbox[b]{\labelwidth}{\hypertarget{#1}{\textit{#1}}:}}
%HEVEA \newcommand{\ruleoneoflabel}[1]{%
%HEVEA   \parbox[b]{\labelwidth}{\hypertarget{#1}{\textit{#1}}: one of}}
%HEVEA \newenvironment{rules}[1]{%
%HEVEA   \begin{list}{}{}%
%HEVEA     \renewcommand{\makelabel}{\rulelabel}%
%HEVEA   \item[#1]}{\end{list}}
%HEVEA \newenvironment{rulesoneof}[1]{%
%HEVEA   \begin{list}{}{}%
%HEVEA     \renewcommand{\makelabel}{\ruleoneoflabel}%
%HEVEA   \item[#1]}{\end{list}}

\newcommand{\nt}[1]{\hyperlink{#1}{\textit{#1}}} % Non-terminal
\newcommand{\tk}[1]{\texttt{#1}}                 % Token (terminal)
\newcommand{\opt}{\ensuremath{_{\it\textcolor{bluegray}{opt}}}}

The grammar of Cyclone is derived from ISO C99.
It has the following additional keywords:
\texttt{abstract}, \texttt{catch},
\texttt{codegen}, \texttt{cut}, \texttt{fallthru}, \texttt{fill},
\texttt{let}, \texttt{malloc},
\texttt{namespace}, \texttt{new}, \texttt{NULL},
\texttt{region_t}, \texttt{regions},
\texttt{rmalloc}, \texttt{rnew}, \texttt{splice},
\texttt{throw}, \texttt{try}, \texttt{tunion}, \texttt{using},
\texttt{xtunion}.  As in gcc,
\texttt{__attribute__} is reserved as well.

The non-terminals
\hypertarget{character-constant}{\textit{character-constant}},
\hypertarget{floating-constant}{\textit{floating-constant}},
\hypertarget{identifier}{\textit{identifier}},
\hypertarget{integer-constant}{\textit{integer-constant}},
\hypertarget{string}{\textit{string}},
\hypertarget{type-var}{\textit{type-var}},
and
\hypertarget{typedef-name}{\textit{typedef-name}}
are defined lexically as in C\@.

The start symbol is \textit{translation-unit}.

\ifscreen
{\footnotesize
\medskip
\noindent
Jump to:~\nt{declaration} $\cdot$
\nt{struct-or-union-specifier} $\cdot$
\nt{tunion-specifier} $\cdot$
\nt{statement} $\cdot$
\nt{expression} $\cdot$
\nt{declarator} $\cdot$
\nt{pattern}

\medskip
}
\fi

\begin{rules}{translation-unit}
  (empty)\\
  \nt{external-declaration} \nt{translation-unit}\opt\\
  \tk{using} \nt{identifier} \tk{;} \nt{translation-unit}\\
  \tk{namespace} \nt{identifier} \tk{;} \nt{translation-unit}\\
  \tk{using} \nt{identifier} {\lb} \nt{translation-unit} {\rb} \nt{translation-unit}\\
  \tk{namespace} \nt{identifier} {\lb} \nt{translation-unit} {\rb} \nt{translation-unit} \\
  \tk{extern} \nt{string} {\lb} \nt{translation-unit} {\rb} \nt{translation-unit}
\end{rules}
\begin{rules}{external-declaration}
  \nt{function-definition}\\
  \nt{declaration}
\end{rules}
\begin{rules}{function-definition}
  \nt{declaration-specifiers}\opt\ \nt{declarator}\\
  \nt{declaration-list}\opt\ \nt{compound-statement}
\end{rules}
\begin{rules}{declaration}
  \nt{declaration-specifiers} \nt{init-declarator-list}\opt\ \tk{;}\\
  \tk{let} \nt{pattern} \tk{=} \nt{expression} \tk{;}\\
  \tk{let} \nt{identifier-list} \tk{;}
\end{rules}
\begin{rules}{declaration-list}
  \nt{declaration}\\
  \nt{declaration-list} \nt{declaration}
\end{rules}
\begin{rules}{declaration-specifiers}
  \nt{storage-class-specifier} \nt{declaration-specifiers}\opt\\
  \nt{type-specifier} \nt{declaration-specifiers}\opt\\
  \nt{type-qualifier} \nt{declaration-specifiers}\opt\\
  \nt{function-specifier} \nt{declaration-specifiers}\opt
\end{rules}
\begin{rulesoneof}{storage-class-specifier}
  \tk{auto} ~
  \tk{register} ~
  \tk{static} ~
  \tk{extern} ~
  \tk{typedef} ~
  \tk{abstract}
\end{rulesoneof}
\begin{rules}{type-specifier}
  \tk{_}\\
  \tk{void}\\
  \tk{char}\\
  \tk{short}\\
  \tk{int}\\
  \tk{long}\\
  \tk{float}\\
  \tk{double}\\
  \tk{signed}\\
  \tk{unsigned}\\
  \nt{enum-specifier}\\
  \nt{struct-or-union-specifier}\\
  \nt{tunion-specifier}\\
  \nt{typedef-name} \nt{type-params}\opt\\
  \nt{type-var}\\
  \nt{type-var} \tk{::} \nt{kind}\\
  \tk{\$(} \nt{parameter-list} \tk{)}\\
  \tk{region_t} \tk{<} \nt{any-type-name} \tk{>}
\end{rules}
\begin{rules}{kind}
  \nt{identifier}\\
  \nt{typedef-name}
\end{rules}
\begin{rulesoneof}{type-qualifier}
  \tk{const} ~
  \tk{restrict} ~
  \tk{volatile}
\end{rulesoneof}
\begin{rules}{enum-specifier}
  \tk{enum} \nt{identifier} {\lb} \nt{enum-declaration-list} {\rb}\\
  \tk{enum} \nt{identifier}
\end{rules}
\begin{rules}{enum-field}
  \nt{identifier}\\
  \nt{identifier} \tk{=} \nt{constant-expression}
\end{rules}
\begin{rules}{enum-declaration-list}
  \nt{enum-field}\\
  \nt{enum-field} \tk{,} \nt{enum-declaration-list}
\end{rules}
\begin{rules}{function-specifier}
  \tk{inline}
\end{rules}
\begin{rules}{struct-or-union-specifier}
  \nt{struct-or-union} {\lb} \nt{struct-declaration-list} {\rb}\\
  \nt{struct-or-union} \nt{identifier} \nt{type-params}\opt\ {\lb} \nt{struct-declaration-list} {\rb}\\
    \nt{struct-or-union} \nt{identifier} \nt{type-params}\opt
\end{rules}
\begin{rules}{type-params}
  \tk{<} \nt{type-name-list} \tk{>}
\end{rules}
\begin{rulesoneof}{struct-or-union}
  \tk{struct} ~
  \tk{union}
\end{rulesoneof}
\begin{rules}{struct-declaration-list}
  \nt{struct-declaration}\\
  \nt{struct-declaration-list} \nt{struct-declaration}
\end{rules}
\begin{rules}{init-declarator-list}
  \nt{init-declarator}\\
  \nt{init-declarator-list} \tk{,} \nt{init-declarator}
\end{rules}
\begin{rules}{init-declarator}
  \nt{declarator}\\
  \nt{declarator} \tk{=} \nt{initializer}
\end{rules}
\begin{rules}{struct-declaration}
  \nt{specifier-qualifier-list} \nt{struct-declarator-list} \tk{;}
\end{rules}
\begin{rules}{specifier-qualifier-list}
  \nt{type-specifier} \nt{specifier-qualifier-list}\opt\\
  \nt{type-qualifier} \nt{specifier-qualifier-list}\opt
\end{rules}
\begin{rules}{struct-declarator-list}
  \nt{struct-declarator}\\
  \nt{struct-declarator-list} \tk{,} \nt{struct-declarator}
\end{rules}
\begin{rules}{struct-declarator}
  \nt{declarator}\\
  \nt{declarator}\opt\ \tk{:} \nt{constant-expression}
\end{rules}
\begin{rules}{tunion-specifier}
  \nt{tunion-or-xtunion} \nt{identifier} \nt{type-params}\opt\ {\lb} \nt{tunionfield-list} {\rb}\\
  \nt{tunion-or-xtunion} \nt{region}\opt\ \nt{identifier} \nt{type-params}\opt\\
  \nt{tunion-or-xtunion} \nt{identifier} \tk{.} \nt{identifier} \nt{type-params}\opt
\end{rules}
\begin{rulesoneof}{tunion-or-xtunion}
  \tk{tunion} ~
  \tk{xtunion}
\end{rulesoneof}
\begin{rules}{tunionfield-list}
  \nt{tunionfield}\\
  \nt{tunionfield} \tk{;}\\
  \nt{tunionfield} \tk{,} \nt{tunionfield-list}\\
  \nt{tunionfield} \tk{;} \nt{tunionfield-list}
\end{rules}
\begin{rulesoneof}{tunionfield-scope}
  \tk{extern} ~
  \tk{static}
\end{rulesoneof}
\begin{rules}{tunionfield}
  \nt{tunionfield-scope} \nt{identifier}\\
  \nt{tunionfield-scope} \nt{identifier} \nt{type-params}\opt\ \tk{(} \nt{parameter-list} \tk{)}
\end{rules}
\begin{rules}{declarator}
  \nt{pointer}\opt\ \nt{direct-declarator}
\end{rules}
\begin{rules}{direct-declarator}
  \nt{identifier}\\
  \tk{(} \nt{declarator} \tk{)}\\
  \nt{direct-declarator} \tk{[} \nt{assignment-expression}\opt\ \tk{]}\\
  \nt{direct-declarator} \tk{(} \nt{parameter-type-list} \tk{)}\\
  \nt{direct-declarator} \tk{(} \tk{;} \nt{effect-set} \tk{)}\\
  \nt{direct-declarator} \tk{(} \nt{identifier-list}\opt\ \tk{)}\\
  \nt{direct-declarator} \tk{<} \nt{type-name-list} \tk{>}
\end{rules}
\begin{rules}{pointer}
  \tk{*}\ \nt{range}\opt\ \nt{region}\opt\ \nt{type-qualifier-list}\opt\ \nt{pointer}\opt\\
  \tk{@}\ \nt{range}\opt\ \nt{region}\opt\ \nt{type-qualifier-list}\opt\ \nt{pointer}\opt\\
  \tk{?}\ \nt{region}\opt\ \nt{type-qualifier-list}\opt\ \nt{pointer}\opt
\end{rules}
\begin{rules}{range}
  {\lb} \nt{assignment-expression} {\rb}
\end{rules}
\begin{rules}{region}
  \tk{_} \\
  \tk{'H}\\
  \nt{type-var}\\
  \nt{type-var} \tk{::} \nt{kind}
\end{rules}
\begin{rules}{type-qualifier-list}
  \nt{type-qualifier}\\
  \nt{type-qualifier-list} \nt{type-qualifier}
\end{rules}
\begin{rules}{parameter-type-list}
  \nt{parameter-list}\\
  \nt{parameter-list} \tk{,} \tk{...}
\end{rules}
\begin{rules}{optional-effect}
  (empty)\\
  \tk{;} \nt{effect-set}
\end{rules}
\begin{rules}{optional-inject}
  (empty)\\
  \nt{identifier}
\end{rules}
\begin{rules}{effect-set}
  \nt{atomic-effect}\\
  \nt{atomic-effect} \tk{+} \nt{effect-set}
\end{rules}
\begin{rules}{atomic-effect}
  {\lb} {\rb}\\
  {\lb} \nt{region-set} {\rb}\\
  \nt{type-var}\\
  \nt{type-var} \tk{::} \nt{kind}
\end{rules}
\begin{rules}{region-set}
  \nt{type-var}\\
  \nt{type-var} \tk{,} \nt{region-set}\\
  \nt{type-var} \tk{::} \nt{kind}\\
  \nt{type-var} \tk{::} \nt{kind} \tk{,} \nt{region-set}
\end{rules}
\begin{rules}{parameter-list}
  \nt{parameter-declaration}\\
  \nt{parameter-list} \tk{,} \nt{parameter-declaration}
\end{rules}
\begin{rules}{parameter-declaration}
  \nt{specifier-qualifier-list} \nt{declarator}\\
  \nt{specifier-qualifier-list} \nt{abstract-declarator}\opt
\end{rules}
\begin{rules}{identifier-list}
  \nt{identifier}\\
  \nt{identifier-list} \tk{,} \nt{identifier}
\end{rules}
\begin{rules}{initializer}
  \nt{assignment-expression}\\
  \nt{array-initializer}
\end{rules}
\begin{rules}{array-initializer}
  {\lb} \nt{initializer-list}\opt\ {\rb}\\
  {\lb} \nt{initializer-list} \tk{,} {\rb}\\
  {\lb} \tk{for} \nt{identifier} \tk{<} \nt{expression} \tk{:} \nt{expression} {\rb}
\end{rules}
\begin{rules}{initializer-list}
  \nt{designation}\opt\ \nt{initializer}\\
  \nt{initializer-list} \tk{,} \nt{designation}\opt\ \nt{initializer}
\end{rules}
\begin{rules}{designation}
  \nt{designator-list} \tk{=}
\end{rules}
\begin{rules}{designator-list}
  \nt{designator}\\
  \nt{designator-list} \nt{designator}
\end{rules}
\begin{rules}{designator}
  \tk{[} \nt{constant-expression} \tk{]}\\
  \tk{.} \nt{identifier}
\end{rules}
\begin{rules}{type-name}
  \nt{specifier-qualifier-list} \nt{abstract-declarator}\opt
\end{rules}
\begin{rules}{any-type-name}
  \nt{type-name}\\
  {\lb} {\rb}\\
  {\lb} \nt{region-set} {\rb}\\
  \nt{any-type-name} \tk{+} \nt{atomic-effect}
\end{rules}
\begin{rules}{type-name-list}
  \nt{type-name}\\
  \nt{type-name-list} \tk{,} \nt{type-name}
\end{rules}
\begin{rules}{abstract-declarator}
  \nt{pointer}\\
  \nt{pointer}\opt\ \nt{direct-abstract-declarator}
\end{rules}
\begin{rules}{direct-abstract-declarator}
  \tk{(} \nt{abstract-declarator} \tk{)}\\
  \nt{direct-abstract-declarator}\opt\ \tk{[} \nt{assignment-expression}\opt\ \tk{]}\\
  \nt{direct-abstract-declarator}\opt\ \tk{(} \nt{parameter-type-list}\opt\ \tk{)}\\
  \nt{direct-abstract-declarator}\opt\ \tk{(} \tk{;} \nt{effect-set} \tk{)}\\
  \nt{direct-abstract-declarator}\opt\ \tk{[} \tk{?} \tk{]}\\
  \nt{direct-abstract-declarator} \tk{<} \nt{type-name-list} \tk{>}
\end{rules}
\begin{rules}{statement}
  \nt{labeled-statement}\\
  \nt{expression-statement}\\
  \nt{compound-statement}\\
  \nt{selection-statement}\\
  \nt{iteration-statement}\\
  \nt{jump-statement}\\
  \tk{region} \nt{identifier} \nt{statement}\\
  \tk{region} \tk{<} \nt{type-var} \tk{>} \nt{identifier} \nt{statement}\\
  \tk{cut} \nt{statement}\\
  \tk{splice} \nt{statement}
\end{rules}
\begin{rules}{labeled-statement}
  \nt{identifier} \tk{:} \nt{statement}
\end{rules}
\begin{rules}{expression-statement}
  \nt{expression}\opt\ \tk{;}
\end{rules}
\begin{rules}{compound-statement}
  {\lb} \nt{block-item-list}\opt\ {\rb}
\end{rules}
\begin{rules}{block-item-list}
  \nt{block-item}\\
  \nt{block-item} \nt{block-item-list}
\end{rules}
\begin{rules}{block-item}
  \nt{declaration}\\
  \nt{statement}
\end{rules}
\begin{rules}{selection-statement}
  \tk{if} \tk{(} \nt{expression} \tk{)} \nt{statement}\\
  \tk{if} \tk{(} \nt{expression} \tk{)} \nt{statement} \tk{else} \nt{statement}\\
  \tk{switch} \tk{(} \nt{expression} \tk{)} {\lb} \nt{switch-clauses} {\rb}\\
  \tk{try} \nt{statement} \tk{catch} {\lb} \nt{switch-clauses} {\rb}
\end{rules}
\begin{rules}{switch-clauses}
  (empty)\\
  \tk{default} \tk{:} \nt{block-item-list}\\
  \tk{case} \nt{pattern} \tk{:} \nt{block-item-list}\opt\ \nt{switch-clauses}\\
  \tk{case} \nt{pattern} \tk{\&\&} \nt{expression} \tk{:} \nt{block-item-list}\opt\ \nt{switch-clauses}
\end{rules}
\begin{rules}{iteration-statement}
  \tk{while} \tk{(} \nt{expression} \tk{)} \nt{statement}\\
  \tk{do} \nt{statement} \tk{while} \tk{(} \nt{expression} \tk{)} \tk{;}\\
  \tk{for} \tk{(} \nt{expression}\opt\ \tk{;} \nt{expression}\opt\ \tk{;} \nt{expression}\opt\ \tk{)} \nt{statement}\\
  \tk{for} \tk{(} \nt{declaration} \nt{expression}\opt\ \tk{;} \nt{expression}\opt\ \tk{)} \nt{statement}
\end{rules}
\begin{rules}{jump-statement}
  \tk{goto} \nt{identifier} \tk{;}\\
  \tk{continue} \tk{;}\\
  \tk{break} \tk{;}\\
  \tk{return} \tk{;}\\
  \tk{return} \nt{expression} \tk{;}\\
  \tk{fallthru} \tk{;}\\
  \tk{fallthru} \tk{(} \nt{argument-expression-list}\opt\ \tk{)} \tk{;}
\end{rules}
\begin{rules}{pattern}
  \tk{_}\\
  \tk{(} \nt{pattern} \tk{)}\\
  \nt{integer-constant}\\
  \tk{-} \nt{integer-constant}\\
  \nt{floating-constant}\\
  \nt{character-constant}\\
  \tk{NULL}\\
  \nt{identifier}\\
  \nt{identifier} \nt{type-params}\opt\ \tk{(} \nt{tuple-pattern-list} \tk{)}\\
  \tk{\$\tk{(}} \nt{tuple-pattern-list} \tk{)}\\
  \nt{identifier} \nt{type-params}\opt\ {\lb} {\rb}\\
  \nt{identifier} \nt{type-params}\opt\ {\lb} \nt{field-pattern-list} {\rb}\\
  \tk{\&} \nt{pattern}\\
  \tk{*} \nt{identifier}
\end{rules}
\begin{rules}{tuple-pattern-list}
  (empty)\\
  \nt{pattern}\\
  \nt{tuple-pattern-list} \tk{,} \nt{pattern}
\end{rules}
\begin{rules}{field-pattern}
  \nt{pattern}\\
  \nt{designation} \nt{pattern}
\end{rules}
\begin{rules}{field-pattern-list}
  \nt{field-pattern}\\
  \nt{field-pattern-list} \tk{,} \nt{field-pattern}
\end{rules}
\begin{rules}{expression}
  \nt{assignment-expression}\\
  \nt{expression} \tk{,} \nt{assignment-expression}
\end{rules}
\begin{rules}{assignment-expression}
  \nt{conditional-expression}\\
  \nt{unary-expression} \nt{assignment-operator} \nt{assignment-expression}
\end{rules}
\begin{rulesoneof}{assignment-operator}
  \tk{=} ~
  \tk{*=} ~
  \tk{/=} ~
  \tk{\%=} ~
  \tk{+=} ~
  \tk{-=} ~
  \tk{<<=} ~
  \tk{>>=} ~
  \tk{\&=} ~
  \tk{\symbol{'136}=} ~
  \tk{|=}
\end{rulesoneof}
\begin{rules}{conditional-expression}
  \nt{logical-or-expression}\\
  \nt{logical-or-expression} \tk{?} \nt{expression} \tk{:} \nt{conditional-expression}\\
  \tk{throw} \nt{conditional-expression} \\
  \tk{new} \nt{array-initializer} \\
  \tk{new} \nt{logical-or-expression} \\
  \tk{rnew} \tk{(} \nt{expression} \tk{)} \nt{array-initializer} \\
  \tk{rnew} \tk{(} \nt{expression} \tk{)} \nt{logical-or-expression}
\end{rules}
\begin{rules}{constant-expression}
  \nt{conditional-expression}
\end{rules}
\begin{rules}{logical-or-expression}
  \nt{logical-and-expression}\\
  \nt{logical-or-expression} \tk{||} \nt{logical-and-expression}
\end{rules}
\begin{rules}{logical-and-expression}
  \nt{inclusive-or-expression}\\
  \nt{logical-and-expression} \tk{\&\&} \nt{inclusive-or-expression}
\end{rules}
\begin{rules}{inclusive-or-expression}
  \nt{exclusive-or-expression}\\
  \nt{inclusive-or-expression} \tk{|} \nt{exclusive-or-expression}
\end{rules}
\begin{rules}{exclusive-or-expression}
  \nt{and-expression}\\
  \nt{exclusive-or-expression}
%HEVEA \begin{rawhtml}<tt>^</tt>\end{rawhtml}
%HEVEA \begin{latexonly}
  \tk{\^}
%HEVEA \end{latexonly}
  \nt{and-expression}
\end{rules}
\begin{rules}{and-expression}
  \nt{equality-expression}\\
  \nt{and-expression} \tk{\&} \nt{equality-expression}
\end{rules}
\begin{rules}{equality-expression}
  \nt{relational-expression}\\
  \nt{equality-expression} \tk{==} \nt{relational-expression}\\
  \nt{equality-expression} \tk{!=} \nt{relational-expression}
\end{rules}
\begin{rules}{relational-expression}
  \nt{shift-expression}\\
  \nt{relational-expression} \tk{<} \nt{shift-expression}\\
  \nt{relational-expression} \tk{>} \nt{shift-expression}\\
  \nt{relational-expression} \tk{<=} \nt{shift-expression}\\
  \nt{relational-expression} \tk{>=} \nt{shift-expression}
\end{rules}
\begin{rules}{shift-expression}
  \nt{additive-expression}\\
  \nt{shift-expression} \tk{<<} \nt{additive-expression}\\
  \nt{shift-expression} \tk{>>} \nt{additive-expression}
\end{rules}
\begin{rules}{additive-expression}
  \nt{multiplicative-expression}\\
  \nt{additive-expression} \tk{+} \nt{multiplicative-expression}\\
  \nt{additive-expression} \tk{-} \nt{multiplicative-expression}
\end{rules}
\begin{rules}{multiplicative-expression}
  \nt{cast-expression}
  \nt{multiplicative-expression} \tk{*} \nt{cast-expression}\\
  \nt{multiplicative-expression} \tk{/} \nt{cast-expression}\\
  \nt{multiplicative-expression} \tk{\%} \nt{cast-expression}
\end{rules}
\begin{rules}{cast-expression}
  \nt{unary-expression}\\
  \tk{(} \nt{type-name} \tk{)} \nt{cast-expression}
\end{rules}
\begin{rules}{unary-expression}
  \nt{postfix-expression}\\
  \tk{++} \nt{unary-expression}\\
  \tk{--} \nt{unary-expression}\\
  \nt{unary-operator} \nt{cast-expression}\\
  \tk{sizeof} \nt{unary-expression}\\
  \tk{sizeof} \tk{(} \nt{type-name} \tk{)}\\
  \nt{expression} \tk{.} \tk{size}
\end{rules}
\begin{rulesoneof}{unary-operator}
  \tk{\&} ~
  \tk{*} ~
  \tk{+} ~
  \tk{-} ~
%HEVEA \begin{rawhtml}<tt>~</tt>\end{rawhtml} ~
%HEVEA \begin{latexonly}
  \tk{\~} ~
%HEVEA \end{latexonly}
  \tk{!}
\end{rulesoneof}
\begin{rules}{postfix-expression}
  \nt{primary-expression}\\
  \nt{postfix-expression} \tk{[} \nt{expression} \tk{]}\\
  \nt{postfix-expression} \tk{(} \tk{)}\\
  \nt{postfix-expression} \tk{(} \nt{argument-expression-list} \tk{)}\\
  \nt{postfix-expression} \tk{.} \nt{identifier}\\
  \nt{postfix-expression} \tk{->} \nt{identifier}\\
  \nt{postfix-expression} \tk{++}\\
  \nt{postfix-expression} \tk{--}\\
  \tk{(} \nt{type-name} \tk{)} {\lb} \nt{initializer-list} {\rb}\\
  \tk{(} \nt{type-name} \tk{)} {\lb} \nt{initializer-list} \tk{,} {\rb}\\
  \tk{fill} \tk{(} \nt{expression} \tk{)}\\
  \tk{codegen} \tk{(} \nt{function-definition} \tk{)}
\end{rules}
\begin{rules}{primary-expression}
  \nt{identifier}\\
  \nt{constant}\\
  \nt{string}\\
  \tk{(} \nt{expression} \tk{)}\\
  \nt{identifier} \tk{<>}\\
  \nt{identifier} \tk{@} \tk{<} \nt{type-name-list} \tk{>}\\
  \tk{\$(} \nt{argument-expression-list} \tk{)}\\
  \nt{identifier} {\lb} \nt{initializer-list} {\rb}\\
  \tk{(} {\lb} \nt{block-item-list} {\rb} \tk{)}
\end{rules}
\begin{rules}{argument-expression-list}
  \nt{assignment-expression}\\
  \nt{argument-expression-list} \tk{,} \nt{assignment-expression}
\end{rules}
\begin{rules}{constant}
  \nt{integer-constant}\\
  \nt{character-constant}\\
  \nt{floating-constant}\\
  \tk{NULL}
\end{rules}

% Local Variables:
% TeX-master: "main-screen"
% End:
